\documentclass[10pt]{article}

\usepackage{german}
\usepackage[official,right]{eurosym}
\usepackage{multicol}

\makeatletter

\def\finalpagebreak{\vfill\pagebreak}

%\usepackage{standard}

\oddsidemargin 0 in      %   Note that \oddsidemargin = \evensidemargin
\evensidemargin 0 in
%\marginparwidth 0.75 in
\textwidth 6.375 true in % Width of text line.
%\textheight 21cm
%\topmargin-1cm

\renewcommand\section[1]{%
 \par\vspace{0.5\baselineskip}%
 \noindent{\bfseries\large #1}%
 \\[0.5\baselineskip]}

\def\EUR{\,\euro}

\def\oldefault{ol}
\DeclareRobustCommand\olshape{\fontshape\oldefault\selectfont}

\setlength\parskip{0.5\baselineskip}
\setlength\parindent{0pt}

\begin{document}
\begin{centering}
\LARGE{The European Currency Symbol \euro{} for \LaTeX}\\
\vspace*{4mm}
\large{by Henrik Theiling}\\
\large{\tt theiling@coli.uni-sb.de}\\
\end{centering}
\vspace*{2\baselineskip}

%======================================================================
\section{Why?}
The European currency symbol \euro{} is already available for \LaTeX{}
in different packages (Text-Companion fonts, Marvosym package,
etc.{}).  However, I wanted to create a symbol that is constructed
according to the official European Commision's definitions.
Furthermore, I wanted to do it with {\sf METAFONT} because I don't
like to use PostScript fonts because they are likely to create
compatibility problems.

\section{Usage}
At the beginning of the document in the pre-amble, declare
\verb:\usepackage{eurosym}:.  Then the new commands
\verb:\officialeuro{}:, \verb:\geneuro{}:, \verb:\geneuronarrow{}: and
\verb:\geneurowide{}: create \officialeuro{}, \geneuro{},
\geneuronarrow{}, and \geneurowide{} resp.  The latter three create an
overlayed symbol from the current font's C and the two horizontal bars
with three different lengths.  Of course you should only use the
latter commands if the font you're using lacks the \officialeuro{}
symbol or you don't like ``\officialeuro{}'' for some reason.
Officially, ``\officialeuro{}'' has to be used with all the fonts
because it's the only official shape.  However, this doesn't always
look nice (especially in bold or oblique font shapes).  Furthermore,
even the OCR draft suggests a different shape for OCR-B.

There is also the command \verb:\euro{}: which defaults to be a
shortcut for \verb:\officialeuro{}:.  You can set a different default
symbol by either declaring e.\,g.{} \verb:\let\euro=\eurogen: or by
using one of the package options {\tt{}official}, {\tt{}gen},
{\tt{}gennarrow} or {\tt{}genwide}.

There is a convenient command \verb:\EUR{:\dots\verb:}: which lets you
typeset an amount of money nicely (with a micro space \verb:\,:
between the symbol and the number).  Because in some countries the
symbol has to appear on the left of the number while in others it has
to be on the right, the packages recognizes the two options {\tt left}
and {\tt right}.  So if you put \verb:\usepackage[left]{eurosym}: at
the beginning of your document, \verb:\EUR{1000}: will create
\euro{}\,1000, while a \verb:\usepackage[right]{eurosym}: makes it
appear as 1000\,\euro{}.  The package default is {\tt [left]} unless
the {\tt german} package was included {\em before\/} the {\tt eurosym}
package.  You can change the shape of the symbol that \verb:\EUR: uses
by redefining \verb:\euro:.

%======================================================================
\section{Table of Commands}
Here is a table of the major commands:

\begin{tabular}{ll}
\verb:\usepackage[:{\it options}\verb:]{eurosym}:
   & include the eurosym package. Available options:\\
   & {\tt left}, {\tt right}, {\tt official}, {\tt gen},
     {\tt gennarrow}, {\tt genwide}.\\
\verb:\euro{}:
   & create a \euro{} symbol.  The shape depends on the \\
   & package options and defaults to \verb:\officialeuro{}:\\
\verb:\EUR{:{\it amount}\verb:}:
   & typeset an amount of \euro{}.  The position of the\\
   & currency symbol depends on the package option and\\
   & defaults to {\tt left} except the {\tt german} package\\
   & is loaded in which case it defaults to {\tt right}.
\end{tabular}

\noindent There should generally be no need to use the following minor
commands.

\begin{tabular}{ll}
\verb:\officialeuro{}:   & create a \officialeuro{} symbol\\
\verb:\geneuro{}:        & create a \geneuro{} symbol\\
\verb:\geneuronarrow{}:  & create a \geneuronarrow{} symbol\\
\verb:\geneurowide{}:    & create a \geneurowide{} symbol\\
\verb:\eurobars{}:       & create only the bars of the symbol: \eurobars\\
\verb:\eurobarsnarrow{}: & create the bars of the symbol in 80\% width: \eurobarsnarrow\\
\verb:\eurobarswide{}:   & create the bars of the symbol in 120\% width: \eurobarswide\\
\end{tabular}

\section{Exact Sizes}
A \euro{} symbol is as tall as a C.  The {\tt bx}-shaped version
should be a little wider than the normal one and should of course be
bold.

\vspace*{2mm}
\setlength\fboxsep{0pt}
\setlength\fboxrule{0.1pt}
\def\EC#1{\def\f@size{#1}\selectfont\let\ff=\f@size{\normalsize\ff\,pt:} \fbox{\euro{}C}}
\begin{tabular}{lll@{\qquad}ll}
{\EC{24.88}} & {\EC{10}} & {\EC{6}} & \tt n  & \noindent{\Huge \fbox{\euro{}}}\\
{\EC{17.28}} & {\EC{9}}  & {\EC{5}} & \tt b  & \noindent{\Huge \fbox{\bfseries\euro}{}}\\
{\EC{14.4}}  & {\EC{8}}  & & &\\
{\EC{12}}    & {\EC{7}}  & & &\\
\end{tabular}


\section{Appearance in Text}
\setlength\parskip{0.5ex}
\begin{tabular}{lll}
\mddefault & \updefault &                    Ich bezahlte 500\EUR{} f\"ur das Radio. Bzw. 1000\,\geneuro{} f\"ur den Fernseher.\\
\mddefault & \sldefault & \slshape           Ich bezahlte 500\EUR{} f\"ur das Radio. Bzw. 1000\,\geneuro{} f\"ur den Fernseher. \\
\mddefault & \itdefault & \itshape           Ich bezahlte 500\EUR{} f\"ur das Radio. Bzw. 1000\,\geneuro{} f\"ur den Fernseher. \\
\mddefault & \scdefault & \scshape           Ich bezahlte 500\EUR{} f\"ur das Radio. Bzw. 1000\,\geneuro{} f\"ur den Fernseher. \\
\bfdefault & \updefault & \bfseries          Ich bezahlte 500\EUR{} f\"ur das Radio. Bzw. 1000\,\geneuro{} f\"ur den Fernseher. \\
\bfdefault & \sldefault & \bfseries\slshape  Ich bezahlte 500\EUR{} f\"ur das Radio. Bzw. 1000\,\geneuro{} f\"ur den Fernseher. \\
\bfdefault & \itdefault & \bfseries\itshape  Ich bezahlte 500\EUR{} f\"ur das Radio. Bzw. 1000\,\geneuro{} f\"ur den Fernseher.
\end{tabular}

\section{Table of Shapes}
The following shapes are derived from the official symbol ``Euro glyph''.\\
\let\testeuro\euro
\begin{tabular}{l|ccc}
              & \tt\updefault=\tt\scdefault & \tt\sldefault=\tt\itdefault & \tt\oldefault \\\hline
\tt\mddefault & \testeuro          & \slshape\testeuro          &{\olshape\officialeuro}\\
\tt\bfdefault & \bfseries\testeuro & \bfseries\slshape\testeuro &{\bfseries\olshape\officialeuro}\\
\end{tabular}

The style file defines \verb:\slshape: as \verb:\itshape: for this symbol and
normal shape for \verb:\scshape:.

\section{Table of Generic Shapes}
The font also contains only the bars for a fast hacking way to create the Euro currency
symbol with fonts that don't contain it.  Usually you can simply use \verb:\geneuro: to
get a hacked Euro symbol for the current font.

\begin{tabular}{l|cc}
              & \tt\updefault      & \tt\sldefault              \\\hline
\tt\mddefault & \geneuro          & \slshape\geneuro          \\
\tt\bfdefault & \bfseries\geneuro & \bfseries\slshape\geneuro
\end{tabular}

If the font you are using is wider or more narrow so that the sizes of
the bars don't look nice for that font, you can either try
\verb:\geneuronarrow: or \verb:\geneurowide:.

\begin{tabular}{l|cccc}
              & \multicolumn{2}{c}{\tt{$\backslash$}geneuronarrow}
              & \multicolumn{2}{c}{\tt{$\backslash$}geneurowide}\\
              & \tt\updefault           & \tt\sldefault
              & \tt\updefault           & \tt\sldefault             \\\hline
\tt\mddefault & \geneuronarrow          & \slshape\geneuronarrow
              & \geneurowide            & \slshape\geneurowide      \\
\tt\bfdefault & \bfseries\geneuronarrow & \bfseries\slshape\geneuronarrow
              & \bfseries\geneurowide   & \bfseries\slshape\geneurowide
\end{tabular}

%======================================================================
\section{Construction of the Symbol}
The construction is taken from the German c't Magazine, 11/98,
page~211.  That construction was missing one measure.  A completion of
my construction was reported by Dr. Werner Gans, who found the full
construction in `Encyclopaedia Britannica, Book of the Year 2002'.

Let the line thickness be $x$.  Then the radius of the inner circle is
$5\,x$ and the distance between the inner bars is $x$.  The angle of
the opening on the right is $80^\circ$.  The $x$-coordinate of the
left pointed end of the bars is $8\,x$ from the center.  All the other
points are obtained by intersection of lines and by parallelism.

\noindent
\setlength\unitlength{1mm}
\begin{picture}(80,80)
\put(30,15)  {\mbox{\fontencoding{U}\fontfamily{eurosym}\def\f@size{200}\selectfont\char0}}
\put(66,38)  {\mbox{$80^\circ$}}
\put(30.3,6) {\mbox{$\longleftarrow\hbox to20pt{~} 8 \times \hbox to20pt{~}\longrightarrow$}}
\put(38.7,10){\mbox{$\longleftarrow\hbox to8pt{~} 6 \times \hbox to8.2pt{~}\longrightarrow$}}
\put(43,38)  {\mbox{$\longleftarrow\hbox to2pt{~} 5 \times \hbox to2pt{~}\longrightarrow$}}
\end{picture}

\section{Example in a Longer Text}
In the following, I've copied an article from a local newspaper
(Neue Westf\"alische, Nr.~174, Donnerstag, 30.~Juli 1998) containing
money amounts and changed ``DM'' to ``\euro'' or ``\geneuro''
resp.{} in order to give an impression of how it looks in a longer
text.

\setlength\premulticols{0pt}
\setlength\postmulticols{0pt}
\def\thetesttext{
\noindent\leftline{\large\bfseries Erzeugergemeinschaft plant bis zum Jahr
2003 Verdopplung des Umsatzes}\\[2ex]
\noindent\leftline{\Large\bfseries EGO will Riesenvorsprung nutzen}
\begin{multicols}{3}
\begin{bfseries}
\noindent B\,i\,s\,s\,e\,n\,d\,o\,r\,f\,/\,L\,a\,g\,e (blo).
Verbraucher kaufen Fleisch- und Wurstwaren
inzwischen sehr kritisch ein, gehen wieder viel h\"aufiger ins
Fleischerfachgesch\"aft.  Das kommt der EGO (Erzeugergemeinschaft
f\"ur Schlachtvieh im Raum Osnabr\"uck e.\,G.{}) mit Ihren
"`Eichenhof"'-Produkten entgegen.  Die EGO setzt auf
nachpr\"ufbare Herkunft und Qualit\"at, kooperiert mit 175 Fleischereien
und plant bis zum Jahr 2003 eine Umsatzverdopplung auf 400 Mio.{}\,\Euro{}.
\end{bfseries}
\par\vspace{1ex}
\noindent Dabei kann die EGO ein gro\ss es Pfund in die Waagschale
werfen. "`Wir haben mindestens 15~Jahre Vorsprung."' erkl\"arte
gesch\"aftsf\"uhrender Vorstandsvorsitzender Karl-Heinz
H\"ugelsmeyer in Bissendort.  Der Vorsprung sind die strengen Kriterien,
nach denen der genossenschaftliche Zusammenschlu\ss{} vor knapp
700~vertraglich gebundenen b\"auerlichen Familienbetrieben zwischen
Teutoburger Wald und Wiehengebirge arbeitet: Tiergerechte Schweine- und
Rinderhaltung mit festen Regeln f\"ur F\"utterung, Zucht und
Aufzucht, Andienungspflicht und Abnahmegarantie, Sauberkeit der Produktion,
Regelverst\"o\ss{}e werden hart mit Ausschlu\ss{} geahndet.
\par\vspace{1ex}
\noindent Die Landwirte profitieren durch gute Auszahlungspreise und
Pr\"amien an die Mitglieder in 1997 aus, berichtete Gesch\"aftsf\"uhrer
Rudolf Fester.  Er l\"ost am 1.~August EGO-Gr\"under Karl-Heinz
H\"uggelsmeyer als Vorstandschef ab.  Der 65j\"ahrige H\"uggelsmeyer
wird noch f\"ur einige Jahre als Gesch\"aftsf\"uhrer der Tochterfirmen
Pieper (Lage) und Kinnius (Osnabr\"uck) t\"atig sein.  diese beiden
Verarbeitsbetriebe erzielen den Angaben zufolge derzeit positivere
Ergebnisse als 1997, weil die Rohstoffpreise sinken.  Insgesamt
stehe die EGO besser da als vor einem Jahr.  F\"ur 1998 rechnet
die 210~Mitarbeiter besch\"aftigende Gruppe mit 200~Mio.{}\,\Euro{},
davon 5~Mio.{}\,\Euro{} mit Convenience-Produkten.  Sie sollen
einmal 15~Mio.{}\,\Euro{} bringen.  1997 war der EGO-Umsatz um
7\% auf 197~Mio.{}\,\Euro{} gestiegen.
\par\vspace{1ex}
\noindent Die Landschlachterei Pieper, die einige ihrer
Abnehmer ausgesiebt hat, kam dabei im Vorjahr auf
29,3 (Vorjahr 30,3) Mio.{}\,\Euro{}\@.  Die Zahl der
Mitgliedsbetriebe stieg um gut 100 auf~687.  Aus der Fusion mit
der Erzeugergemeinschaft Minden-Ravensberg-Lippe (Herford),
die 320~Mitglieder hatte, kamen nur 81~Betriebe hinzu.  Das Gros
wurde nicht \"ubernommen.  Zitat: "`Die wollten unsere Kriterien
nicht erf\"ullen."'
\end{multicols}
}

\sloppy
\vfill\pagebreak
\let\Euro\euro
\thetesttext

\vfill\pagebreak
\let\Euro\geneuro
\thetesttext

\end{document}
